Table 1

Ages of nodes extracted from 9 chronograms containing at least 2 of the 6 bird species used in the Fringillidae small example of the DateLife workflow. All nodes are congruent with the chosen tree topology. See Figure 3, step c2.

Table 2

Summary statistics of node ages from Table 1. Median ages are used by default to date a tree topology using DateLife, see Figure 3, step c3.

Figure 1

Main DateLife workflow to obtain sceintific data of time of lineage divergence for a set of taxon names. a) The workflow starts with at least two taxon names provided by the user as a list or as tip names on a tree. Input names can contain synonyms and misspellings.
b) Taxon names are processed using the Taxonomic Name Resolution Service (TNRS) and are standardized to the taxonomy of the chronogram database. In this mock example, 4 names (in bold) are synonyms in the standardized taxonomy.
c) Processed taxon names are searched in the chronogram database. Chronograms that have at least two taxon names are identified (*), pruned down to matching taxon names only, and stored as source chronograms for the set of input taxon names.
d) A topology to map ages from source chronograms can be provided by the user. If no topology is provided, OpenTree's synthetic tree topology is used. Alternative topologies that can be used are the largest (with the most input taxon names) source chronogram or a phylogeny constructed using genetic data from the Barcode of Life Database (BOLD).
e) Nodes from source chronogram are congruified to nodes from the tree topology chosen in step (d). In this example all ages from source chronograms are congruent with nodes on the tree topology and provide time of divergenve information for all but one node (n2).
f) Source node ages per node are summarized.
g) Summary node ages are used as secondary calibrations to date the chosen tree topology. In this example we show the chronogram of median ages.
All this analyses can be performed via DateLife's interactive website at \url{www.datelife.org}, or using the \texttt{datelife} R package. Details on the R functions used to perform the analyses are available from \texttt{datelife}'s R package vignettes at \url{https://phylotastic.org/datelife}.

Figure 2:

DateLife's benchmarking results. Computation time used to process a query and a search across \texttt{datelife}'s chronogram database, relative to number of input taxon names. For each N = \{10, 100, 200, ..., 1 000, ... , 9 000, 10 000\}, we sampled N species names from the class Aves a hundred times, and then performed a \texttt{datelife} search processing the input names using the Taxon Names Resolution Service (TNRS; light gray), and without processing names (dark gray). For comparison, we performed a search using an input that has been pre-processed with TNRS (light blue).

Figure 3:

DateLife analysis results for a small sample of 6 bird species within the Passeriformes (a). Processed species names were found across 9 chronograms within 6 independent studies (b; Barker et al. (\protect\hyperlink{ref-barker2012going}{2012}), Barker et al. (\protect\hyperlink{ref-barker2015new}{2015}), Burns et al. (\protect\hyperlink{ref-burns2014phylogenetics}{2014}), Hedges et al. (\protect\hyperlink{ref-Hedges2015}{2015}), Hooper and Price (\protect\hyperlink{ref-hooper2017chromosomal}{2017}), Jetz et al. (\protect\hyperlink{ref-Jetz2012}{2012}). This revealed 28 source age data points for the queried species names. Summarized age data was used as secondary calibrations to date a tree topology obtained from OpenTree's synthetic tree v13.4, resulting in a chronogram of summary source ages (c).


Figure 4:

Tree topologies extracted from Open Tree of Life's (OpenTree) synthetic phylogenetic tree.
Topology of 2,333 tips and 1,305 internal nodes (a), encompassing bird species within the family Fringillidae following the NCBI taxonomy (black), as well as all other bird species that share the same Most Recent Common Ancestor (MRCA) node in OpenTree's synthetic tree (purple).
(Subtree topology of 289 tips and 253 internal nodes (b), resulting from pruning species that do not belong to the family Fringillidae according to the NCBI taxonomy (purple branches in topology a).
Bird species within the Fringillidae are paraphyletic
(Alström et al. \protect\hyperlink{ref-alstrom2014discovery}{2014},
Barker, Cibois, Schikler, Feinstein, \& Cracraft \protect\hyperlink{ref-barker2004phylogeny}{2004},
Barker et al. \protect\hyperlink{ref-barker2013going}{2013},
Barker \protect\hyperlink{ref-barker2014mitogenomic}{2014},
Barker et al. \protect\hyperlink{ref-barker2015new}{2015},
Beresford, Barker, Ryan, \& Crowe \protect\hyperlink{ref-beresford2005african}{2005},
Bryson Jr et al. \protect\hyperlink{ref-bryson2014diversification}{2014},
Burleigh, Kimball, \& Braun \protect\hyperlink{ref-burleigh2015building}{2015},
Burns et al. \protect\hyperlink{ref-burns2014phylogenetics}{2014},
Chaves, Hidalgo, \& Klicka \protect\hyperlink{ref-chaves2013biogeography}{2013},
Claramunt \& Cracraft \protect\hyperlink{ref-claramunt2015new}{2015},
Gibb et al. \protect\hyperlink{ref-gibb2015new}{2015},
Hackett et al. \protect\hyperlink{ref-hackett2008phylogenomic}{2008},
Jetz et al. \protect\hyperlink{ref-Jetz2012}{2012},
Johansson, Fjeldså, \& Bowi \protect\hyperlink{ref-johansson2008phylogenetic}{200},
Kimball et al. \protect\hyperlink{ref-kimball2019phylogenomic}{2019},
Klicka et al. \protect\hyperlink{ref-klicka2014comprehensive}{2014},
Lamichhaney et al. \protect\hyperlink{ref-lamichhaney2015evolution}{2015},
Lerner, Meyer, James, Hofreiter, \& Fleischer \protect\hyperlink{ref-lerner2011multilocus}{2011},
Lovette et al. \protect\hyperlink{ref-lovette2010comprehensive}{2010},
Moyle et al. \protect\hyperlink{ref-moyle2016tectonic}{2016},
Ödeen, Håstad, \& Alström \protect\hyperlink{ref-odeen2011evolution}{2011},
Oliveros et al. \protect\hyperlink{ref-oliveros2019earth}{2019},
Päckert et al. \protect\hyperlink{ref-packert2012horizontal}{2012},
Parchman, Benkman, \& Mezquida \protect\hyperlink{ref-parchman2007coevolution}{2007},
Powell et al. \protect\hyperlink{ref-powell2014comprehensive}{2014},
Price et al. \protect\hyperlink{ref-price2014niche}{2014},
Pulgarín-R, Smith, Bryson Jr, Spellman, \& Klicka \protect\hyperlink{ref-pulgarin2013multilocus}{2013},
Selvatti, Gonzaga, \& Moraes Russo \protect\hyperlink{ref-selvatti2015paleogene}{2015},
Tietze, Päckert, Martens, Lehmann, \& Sun \protect\hyperlink{ref-tietze2013complete}{2013},
Treplin et al. \protect\hyperlink{ref-treplin2008molecular}{2008},
Zuccon, Prŷs-Jones, Rasmussen, \& Ericson \protect\hyperlink{ref-zuccon2012phylogenetic}{2012}).

Figure 5:

Fringillidae median summary chronogram generated with DateLife. The original topology has 289 tips and 253 nodes, from which 194 have age data from at least one published chronogram. In total, 19 different chronograms from 13 different studies contribute 818 age data points, which were summarized to obtain a single value for each one of the 194 nodes with age data. From the 194 summary ages available, 21 were not used as calibrations (asterisk, *), because they were older than a parent node or younger than a descendant node; the remaining 173 summary ages were used as secondary calibrations (forward slash, /) with the Branch Length Adjuster (BLADJ) software from Webb et al., (\protect\hyperlink{ref-Webb2008}{2008}).



Figure 6:

Cross validation of results from a DateLife analysis of the family Fringillidae, shown in Fig. 5. Each plot compares node age estimate from an original study chronogram (x axis) with the corresponding node age obtained with a DateLife analysis (y axis).
