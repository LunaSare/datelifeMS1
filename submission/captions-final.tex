Table 1

Node ages extracted with DateLife from 9 chronograms (shown in Fig. 2c) containing at least 2 of the 6 bird species analysed in the Fringillidae small example. All nodes are congruent with the chosen tree topology shown in Figure 2d.

Table 2

Summary statistics of node ages from Table 1. DateLife uses median ages by default to date a chosen tree topology (Fig. 3g).

Figure 1:

DateLife's benchmarking results showing computation time used for taxon name processing and search across \texttt{datelife}'s chronogram database, as a function of number of input taxon names (N). For each N = \{10, 100, 200, ..., 1 000, ... , 9 000, 10 000\}, we randomly sampled N species names from the class Aves, a hundred times, and then performed a \texttt{datelife} search processing the input names using the Taxon Names Resolution Service (TNRS; light gray), and without processing input names (dark gray). For comparison, we performed a chronogram search using names that have been pre-processed with TNRS (light blue).

Figure 2:

DateLife results of an analysis of a small sample of 6 bird species within the Passeriformes (a, b). Processed species names were found across 9 chronograms within 6 independent studies (c; Barker et al. (\protect\hyperlink{ref-barker2012going}{2012}), Barker et al. (\protect\hyperlink{ref-barker2015new}{2015}), Burns et al. (\protect\hyperlink{ref-burns2014phylogenetics}{2014}), Hedges et al. (\protect\hyperlink{ref-Hedges2015}{2015}), Hooper and Price (\protect\hyperlink{ref-hooper2017chromosomal}{2017}), Jetz et al. (\protect\hyperlink{ref-Jetz2012}{2012}). This revealed 28 source age data points for the queried species names (e; Table 1). Summarized age data (f; Table 2) was used as secondary calibrations to date a tree topology obtained from OpenTree's synthetic tree v13.4 (d), resulting in the chronogram of summary source ages shown in (g).




Figure 3:

Median summary chronogram resulting from a Datelife analysis of bird species within the family Fringillidae. For visualization purposes, we are showing a portion of the final median summary chronogram encompassing 57 species out of the 289 total included in the analysis. The complete chronogram is available as Supplementary Figure S3. The working tree topology has 289 tip species and 253 nodes, from which 194 have age data from at least one published chronogram. In total, 19 different chronograms from 13 different studies contributed 818 age data points, which were summarized to obtain a single value for each one of the 194 nodes with age data. From the 194 summary ages available, 21 were discarded and not used as calibrations (asterisk, *), because they were older than a parent node or younger than a descendant node; the remaining 173 summary ages were used as secondary calibrations (forward slash, /) with the Branch Length Adjuster (BLADJ) software from Webb et al., (\protect\hyperlink{ref-Webb2008}{2008}).



Figure 4:

Cross validation of results from a DateLife analysis of the family Fringillidae, shown in Fig. 3 and Supplementary Figure S3. Each plot compares original node age estimates from an input source study chronogram (x axis) with the corresponding node age resulting from a DateLife analysis excluding data from that study (y axis).
