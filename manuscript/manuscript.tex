% Options for packages loaded elsewhere
\PassOptionsToPackage{unicode}{hyperref}
\PassOptionsToPackage{hyphens}{url}
%
\documentclass[
  english,
  man]{apa6}
\usepackage{lmodern}
\usepackage{amssymb,amsmath}
\usepackage{ifxetex,ifluatex}
\ifnum 0\ifxetex 1\fi\ifluatex 1\fi=0 % if pdftex
  \usepackage[T1]{fontenc}
  \usepackage[utf8]{inputenc}
  \usepackage{textcomp} % provide euro and other symbols
\else % if luatex or xetex
  \usepackage{unicode-math}
  \defaultfontfeatures{Scale=MatchLowercase}
  \defaultfontfeatures[\rmfamily]{Ligatures=TeX,Scale=1}
\fi
% Use upquote if available, for straight quotes in verbatim environments
\IfFileExists{upquote.sty}{\usepackage{upquote}}{}
\IfFileExists{microtype.sty}{% use microtype if available
  \usepackage[]{microtype}
  \UseMicrotypeSet[protrusion]{basicmath} % disable protrusion for tt fonts
}{}
\makeatletter
\@ifundefined{KOMAClassName}{% if non-KOMA class
  \IfFileExists{parskip.sty}{%
    \usepackage{parskip}
  }{% else
    \setlength{\parindent}{0pt}
    \setlength{\parskip}{6pt plus 2pt minus 1pt}}
}{% if KOMA class
  \KOMAoptions{parskip=half}}
\makeatother
\usepackage{xcolor}
\IfFileExists{xurl.sty}{\usepackage{xurl}}{} % add URL line breaks if available
\IfFileExists{bookmark.sty}{\usepackage{bookmark}}{\usepackage{hyperref}}
\hypersetup{
  pdftitle={DateLife: leveraging databases and analytical tools to reveal the dated Tree of Life},
  pdfauthor={Luna L. Sánchez Reyes1,2, Emily Jane McTavish1, \& Brian O'Meara2},
  pdflang={en-EN},
  pdfkeywords={Tree; Phylogeny; Scaling; Dating; Ages; Divergence times; Open Science; Congruification; Supertree; Calibrations},
  hidelinks,
  pdfcreator={LaTeX via pandoc}}
\urlstyle{same} % disable monospaced font for URLs
\usepackage{graphicx,grffile}
\makeatletter
\def\maxwidth{\ifdim\Gin@nat@width>\linewidth\linewidth\else\Gin@nat@width\fi}
\def\maxheight{\ifdim\Gin@nat@height>\textheight\textheight\else\Gin@nat@height\fi}
\makeatother
% Scale images if necessary, so that they will not overflow the page
% margins by default, and it is still possible to overwrite the defaults
% using explicit options in \includegraphics[width, height, ...]{}
\setkeys{Gin}{width=\maxwidth,height=\maxheight,keepaspectratio}
% Set default figure placement to htbp
\makeatletter
\def\fps@figure{htbp}
\makeatother
\setlength{\emergencystretch}{3em} % prevent overfull lines
\providecommand{\tightlist}{%
  \setlength{\itemsep}{0pt}\setlength{\parskip}{0pt}}
\setcounter{secnumdepth}{-\maxdimen} % remove section numbering
% Make \paragraph and \subparagraph free-standing
\ifx\paragraph\undefined\else
  \let\oldparagraph\paragraph
  \renewcommand{\paragraph}[1]{\oldparagraph{#1}\mbox{}}
\fi
\ifx\subparagraph\undefined\else
  \let\oldsubparagraph\subparagraph
  \renewcommand{\subparagraph}[1]{\oldsubparagraph{#1}\mbox{}}
\fi
% Manuscript styling
\usepackage{upgreek}
\captionsetup{font=singlespacing,justification=justified}

% Table formatting
\usepackage{longtable}
\usepackage{lscape}
% \usepackage[counterclockwise]{rotating}   % Landscape page setup for large tables
\usepackage{multirow}		% Table styling
\usepackage{tabularx}		% Control Column width
\usepackage[flushleft]{threeparttable}	% Allows for three part tables with a specified notes section
\usepackage{threeparttablex}            % Lets threeparttable work with longtable

% Create new environments so endfloat can handle them
% \newenvironment{ltable}
%   {\begin{landscape}\centering\begin{threeparttable}}
%   {\end{threeparttable}\end{landscape}}
\newenvironment{lltable}{\begin{landscape}\centering\begin{ThreePartTable}}{\end{ThreePartTable}\end{landscape}}

% Enables adjusting longtable caption width to table width
% Solution found at http://golatex.de/longtable-mit-caption-so-breit-wie-die-tabelle-t15767.html
\makeatletter
\newcommand\LastLTentrywidth{1em}
\newlength\longtablewidth
\setlength{\longtablewidth}{1in}
\newcommand{\getlongtablewidth}{\begingroup \ifcsname LT@\roman{LT@tables}\endcsname \global\longtablewidth=0pt \renewcommand{\LT@entry}[2]{\global\advance\longtablewidth by ##2\relax\gdef\LastLTentrywidth{##2}}\@nameuse{LT@\roman{LT@tables}} \fi \endgroup}

% \setlength{\parindent}{0.5in}
% \setlength{\parskip}{0pt plus 0pt minus 0pt}

% \usepackage{etoolbox}
\makeatletter
\patchcmd{\HyOrg@maketitle}
  {\section{\normalfont\normalsize\abstractname}}
  {\section*{\normalfont\normalsize\abstractname}}
  {}{\typeout{Failed to patch abstract.}}
\patchcmd{\HyOrg@maketitle}
  {\section{\protect\normalfont{\@title}}}
  {\section*{\protect\normalfont{\@title}}}
  {}{\typeout{Failed to patch title.}}
\makeatother
\shorttitle{DateLife: revealing the dated Tree of Life}
\keywords{Tree; Phylogeny; Scaling; Dating; Ages; Divergence times; Open Science; Congruification; Supertree; Calibrations\newline\indent Word count: 2628}
\DeclareDelayedFloatFlavor{ThreePartTable}{table}
\DeclareDelayedFloatFlavor{lltable}{table}
\DeclareDelayedFloatFlavor*{longtable}{table}
\makeatletter
\renewcommand{\efloat@iwrite}[1]{\immediate\expandafter\protected@write\csname efloat@post#1\endcsname{}}
\makeatother
\usepackage{lineno}

\linenumbers
\usepackage{csquotes}
\ifxetex
  % Load polyglossia as late as possible: uses bidi with RTL langages (e.g. Hebrew, Arabic)
  \usepackage{polyglossia}
  \setmainlanguage[]{english}
\else
  \usepackage[shorthands=off,main=english]{babel}
\fi

\title{DateLife: leveraging databases and analytical tools to reveal the dated Tree of Life}
\author{Luna L. Sánchez Reyes\textsuperscript{1,2}, Emily Jane McTavish\textsuperscript{1}, \& Brian O'Meara\textsuperscript{2}}
\date{}


\authornote{

School of Natural Sciences, University of California, Merced, Science and Engineering Building 1.

Department of Ecology and Evolutionary Biology, University of Tennessee, Knoxville, 425 Hesler Biology Building, Knoxville, TN 37996, USA.

The authors made the following contributions. Luna L. Sánchez Reyes: Data curation, Investigation, Software, Visualization, Validation, Writing - Original Draft Preparation, Writing - Review \& Editing; Emily Jane McTavish: Resources, Software, Writing - Review \& Editing; Brian O'Meara: Conceptualization, Funding acquisition, Methodology, Resources, Software, Supervision, Writing - Review \& Editing.

Correspondence concerning this article should be addressed to Luna L. Sánchez Reyes, . E-mail: \href{mailto:sanchez.reyes.luna@gmail.com}{\nolinkurl{sanchez.reyes.luna@gmail.com}}

}

\affiliation{\vspace{0.5cm}\textsuperscript{1} University of California, Merced\\\textsuperscript{2} University of Tennessee, Knoxville}

\abstract{
Time of evolutionary origin is fundamental for research in the natural sciences, as well as for education, science communication and policy.
Despite an increased availability of fossil and molecular data, and time-efficient analytical techniques, achieving a high-quality reconstruction of time of evolutionary origin as a phylogenetic tree with branch lengths proportional to absolute time (chronogram), is still a difficult and time-consuming task for a majority of interested parties.
Yet, the amount of published chronograms has increased significantly in the past two decades, and a non-negligeable proportion of these data have been steadily accumulating in public, open databases such as TreeBASE and Open Tree
of Life, exposing a wealth of expertly-curated and peer-reviewed data on time of evolutionary origin in a programatic and reusable way, for a large quantity and diversity of organisms.
This trend results from intensive and localized efforts for improving data sharing practices, as well as incentivizing open science in biology. Despite these trends, accessibility for time data is not that good.
R has become a widely used element of the biological data analyst toolkit. Hence,to improve accessibility of time data we developed an R package that access these data and eases interaction, reanalysis and reuse of it, incorporation of these data into the evolutionary workflow.

Here we present \texttt{datelife}, a service implemented as an R package and a web site
(www.datelife.org) for increased accessibility, efficient reuse, summary and reanalysis of expert, peer-reviewed, public data on time of evolutionary origin.

Main results:
1. blah
2. blah
3. blah

Results in a general context:
All source and summary chronograms can be saved in formats that permit easy reuse and reanalysis. Summary and newly generated trees are potentially useful to evaluate evolutionary hypothesis in different areas of research in biology.
How well this trees work for this purpose still needs to be tested.

\texttt{datelife} will be useful to increase awereness on the existing variation in expert time of divergence data, and might foster exploration of the effect of alternative divergence time hypothesis on the results of analyses, nurturing a culture of more cautious interpretation of evolutionary results.
}



\begin{document}
\maketitle

\hypertarget{introduction}{%
\section{Introduction}\label{introduction}}

\hypertarget{implementationdescriptionworkflow}{%
\section{Implementation/Description/Workflow}\label{implementationdescriptionworkflow}}

The main goal of \texttt{datelife} is to generate a chronogram for any given combination of taxon names, based on expert scientific information.

The \texttt{datelife} workflow builds off of functions from several R packages (rotl (Michonneau, Brown, \& Winter, 2016), ape (Paradis, Claude, \& Strimmer, 2004),
geiger (Harmon, Weir, Brock, Glor, \& Challenger, 2008), paleotree (Bapst, 2012), bold (Chamberlain, 2018), phytools (Revell, 2012), taxize (Chamberlain, 2018; Chamberlain \& Szöcs, 2013), phyloch (Heibl, 2008), and phylocomr (Ooms \& Chamberlain, 2018)).

The basic \texttt{datelife} workflow is shown in figure \ref{fig:workflow}, largely:

\begin{enumerate}
\item It starts with an input consisting of at least two taxon names, which can be provided as a comma separated character string, or as tip labels on a tree. The tree can be provided in newick format, also as a character string, or as a "phylo" R object, and can have any type of branch lengths or none. 
\item The input taxon names are cleaned with TNRS and saved as a `datelifeQuery` object. If taxon names are taxonomic groups above the species level, `datelife` has two alternative behaviors. If the "get species from taxon" flag is active, `datelife` will retrieve all species within a higher taxon name and add the species names to the input. If the flag is inactive, `datelife` will drop the higher taxon names from the input. The cleaned input taxon names are searched across the source chronogram database. Source chronograms with at least two matching input taxon names are singled out and pruned down to preserve only input taxon names in the tips of the chronograms. 
<!-- The resulting pruned source chronograms are hereafter referred as source chronograms.  --> <!-- This was generating some confusion, so I'm calling them pruned source chronograms now-->
Then, each pruned source chronogram is transformed to a patristic distance matrix. This format facilitates and greatly speeds up all downstream analyses and summaries. The matrices are associated to the citation of the original study and stored as a `datelifeResult` object.
\item  At this point, various summary data can be obtained to inform decisions for the next steps of the analysis workflow. Types of summary information provided are: a) all pruned source chronograms, b) age of the MRCA (most recent common ancestor) of the pruned source chronograms, c) citations of studies where pruned source chronograms were originally published, d) a summary table with all of the above, e) a single summary chronogram of all or a subset of pruned source chronograms, f) a report of successful matches of input taxon names across pruned source chronograms, and g) the single pruned source chronogram with the most matching input taxon names.
\item  Finally, time of lineage divcergence obtained from the pruned source chronograms can be used as secondary calibration points to date a tree with or without branch lengths containing some or all input taxon names. %<!--, a taxonomic tree-->
\item  If there is no information available for any input taxon name, users can also create both age and phylogenetic data for the missing branches with a variety of algorithms described below.
\item  Users can easily save all source and summary chronograms in formats that permit easy reuse and reanalyses (newick and R "phylo" format), as well as view and compare results graphically, or construct their own graphs using \texttt{datelife}'s graphic generation functions.
\end{enumerate}

\hypertarget{benchmark}{%
\section{Benchmark}\label{benchmark}}

\texttt{datelife}'s code speed was tested on an Apple iMac
with one 3.4 GHz Intel Core i5 processor.
We registered variation in computing time of query processing and search through the database relative to number of queried taxon names.
Query processing time increases roughly linearly with number of input taxon names, and
increases considerably if the taxonomic name resolution service (TNRS; Boyle et al., 2013) is activated.
Up to ten thousand names can be processed and searched in less than 30 minutes with the most time consuming settings.
Once names have been processed as described in methods, a name search through the chronogram database can be performed in less than a minute, even with a very large number of taxon names (Fig. \ref{fig:runtime1}).
\texttt{datelife}'s code performance was evaluated with a set of unit tests designed and
implemented with the R package testthat (R Core Team, 2018) that were run both locally
with the devtools package (R Core Team, 2018), and on a public server --via
GitHub, using the continuous integration tool Travis CI (\url{https://travis-ci.org}). At
present, unit tests cover more than 30\% of \texttt{datelife}'s code (\url{https://codecov.io/gh/phylotastic/datelife}).

\hypertarget{results}{%
\section{Results}\label{results}}

\hypertarget{case-study}{%
\subsection{Case study}\label{case-study}}

We illustrate the \texttt{datelife} workflow using the family of true finches, Fringillidae as an example. To contextualize, a college educator wishes to know the state-of-the-art on time of evolutionary origin of species belonging to the true finches using \texttt{datelife}.
One option is to go to the website at www.datelife.org and perform an interactive run. However, the educator wants the students to practice their R skills.
The first step is to run a higher-taxon-name \texttt{datelife} query. This will get taxon names for all recognised species within any higher taxon. The Fringillidae has 289 species, according to the Open Tree of Life taxonomy.
Once with a curated set of query taxon names, the next step is to run a \texttt{datelife} search. This will find all chronograms that contain at least two queried taxon names, and will save the information on time of lineage divergence as (an R \enquote{data frame}) table.
There are 13 chronograms containing at least two Fringillidae species, published in 9 different studies (Fig. \ref{fig:schronograms1}).
The final step is to summarize the available information using the two alternative types of summary chronograms, median and SDM. As explained in the \enquote{Description} section, data from source chronograms is first summarised into a single distance matrix (using the median and the SDM method respectively) and then the available node ages are used as fixed ages over a consensus tree topology, to obtain a fully dated tree with the program BLADJ (Fig. \ref{fig:summaries}). Median summary chronograms are older and have wider variation in maximum ages than chronograms obtained with SDM. With both methods, ages are generally consistent with source ages, but there are some biological examples in which this is not true (see Discussion).

\hypertarget{cross-validation-test}{%
\subsection{Cross-validation test}\label{cross-validation-test}}

Data from source chronograms can be also used to date tree topologies with no branch lengths, as well as trees with branch lengths as relative substitution rates (Figs. \ref{fig:cvbladj} and \ref{fig:cvbold}). As a form of cross validation, we took tree topologies from each study and calibrated them using time of lineage divergence data from all other source chronograms. In the absence of branch lengths, the ages of internal nodes were recovered with a high precision in almost all cases (except for studies 3, and 5; Fig. \ref{fig:cvbladj}).
Maximum tree ages were only recovered in one case (study 2; Fig. \ref{fig:cvbladj}).
We also demonstrate the usage of PATHd8 (Britton, Anderson, Jacquet, Lundqvist, \& Bremer, 2007) as an alternative method to BLADJ. For this, we run a \texttt{datelife} branch length reconstruction that searches for DNA sequence data from the Barcode of Life Data System {[}BOLD; ratnasingham2007bold{]} to generate branch lengths. We were able to successfully generate a tree with BOLD branch lengths for all of the Fringillidae source chronograms. However, dating with PATHd8 using congruified calibrations, was only successful in
three cases (studies 3, 5, and 9, shown in Fig. \ref{fig:cvbold}). From these, two trees have a different sampling than the original source chronogram, mainly because DNA BOLD data for some species is absent from the database. Maximum ages are quite different from source chronograms, but this might be explained also by the differences in sampling between source chronograms and BOLD trees.
More examples and code used to generate these trees were developed on an open repository that is available for consultation and reuse at \url{https://github.com/LunaSare/datelife_examples}.

\hypertarget{discussion}{%
\section{Discussion}\label{discussion}}

The main goal of \texttt{datelife} is to make expert information on time of lineage divergence easily accesible for comparison, reuse, and reanalysis, to researchers in all areas of science and with all levels of expertise in the matter. It is a very fast tool that fulfills the quality of openness and does not require any expert biological knowledge from users --besides the names of the organisms they want to work with-- for any of its functionalities. However, it has many flaws. Some of them can be overcome, some of them might represent limitations.

Up to the time this manuscript was written, \texttt{datelife}'s chronogram database had 231 chronograms, pulled entirely from OpenTree's tree repository, the only public tree repository from where \texttt{datelife} can currently get chronograms to construct its database. This represents 5.79\% of the largest existing chronogram database, TimeTree, which has a collection of 3,998 chronograms as of November 01 2021. Unfortunately, TimeTree's database is not open for scientific reuse nor automatised data mining (Kumar, Stecher, Suleski, \& Hedges, 2017).
In 2015, a synthetic chronogram was constructed from 2,274 chronograms available at the time on the TimeTree database (Hedges, Marin, Suleski, Paymer, \& Kumar, 2015). This is the only synthetic TimeTree chronogram that has been made publicly available and deposited on the OpenTree repository, and is part of datelife's database now. Hence, the amount of lineages represented in datelife's database is at least as substantial as TimeTree's, ensuring that some information will be available for any given taxon or lineage. Regrettably, this does not ensure that the full state of knowledge of time of divergence of the taxon/lineage will be available.
Incorporation of more published chronograms into \texttt{datelife}'s database is crucial to improve its services. One option to increase our database is the Dryad data repository. Methods to automatically mine chronograms from Dryad could be designed and implemented. However, Dryad's metadata system has no information to automatically detect branch length units, and those would still need to be determined on a second step, by a curator.
Consequently, we would like to emphasize on the importance of sharing chronogram data for the benefit of the scientific community as a whole, into repositories that require expert input and manual curation, such as OpenTree's tree repository (McTavish et al., 2015).

Another potential concern comes from summary chronograms. We currently summarize by default all source chronograms that overlap with at least two taxa. Users can subset source data if they have reasons to choose some source chronograms over others. Strictly speaking, a good chronogram should reflect the real time of lineage divergence accurately and precisely. To our knowledge, there is no objective way to determine if an expert chronogram is better than another. Some criteria that have been put forward are the level of lineage sampling and the number of calibrations used. Scientists usually also favor chronograms constructed using primary calibrations (ages obtained from the fossil or geological record) to ones constructed with secondary calibrations (ages coming from other chronograms). It has been observed with simulations that divergence times inferred with secondary calibrations are significantly younger than those inferred with primary calibrations in analyses performed with bayesian inference methods when priors are implemented in similar ways in both analyses (Schenk, 2016). Yet, there are different ways to use secondary calibrations
and that same bias might not be encountered with dating methods that do not require setting priors, i.e., Maximum Likelihood methods such as r8s (Sanderson, 2003). Certainly, further studies are required to fully understand the effect of using secondary calibrations on time estimates and downstream anlyses.

Furthermore, even chronograms obtained with primary fossil data can show substantial variation in time estimates between clades, as observed from the comparison of source chronograms in the Fringillidae example.
This observation is often encountered in the literature
(see, for example, the ongoing debate about crown group age of angiosperms (Barba-Montoya, Reis, Schneider, Donoghue, \& Yang, 2018; Magallón, Gómez-Acevedo, Sánchez-Reyes, \& Hernández-Hernández, 2015; Ramshaw et al., 1972; Sanderson \& Doyle, 2001). For some studies, especially ones based on branch lengths (e.g., studies of species diversification, timing of evolutionary events, phenotypic trait evolution), using a different chronogram may return different results (Title \& Rabosky, 2016). Stitching together these chronograms can create a larger tree that uses information from multiple studies, but the effect of uncertainties and errors here on downstream analyses is still largely unknown.

Summarizing chronograms might also imply summarizing fundamentally distinct evolutionary hypotheses. For example, two different researchers working on the same clade both carefully select and argument their choices of fossil calibrations. Still, if one researcher decides a fossil will calibrate the ingroup of a clade, while another researcher uses teh same one to calibrate outside the clade, the resulting age estimates will probably differ substantially (the placement of calibrations is proved to deeply affect estimated times of lineage divergence). Trying to summarize the resulting chronograms into a single one using simple summary statistics might erase all types of relevant information from the source chronograms. Accordingly, the prevailing view in our research community is that we should favor time of lineage divergence estimates obtained from a single analysis, using fossil data as primary sources of calibrations, and using fossils that have been widely discussed and curated as calibrations to date other trees, making sure that all data used in the analysis reflect a coherent evolutionary history (Antonelli et al., 2017).
However, the exercise of summarizing different chronograms ha sthe potential to help getting a single global evolutionary history for a lineage by putting together evidence from different hypothesis. Choosing the elements of the chronograms that we are going tp keep and the ones that we are going to discard is key, since we are potentially loosing important parts of the evolutionary history of a lineage that might only be reflected in source chronograms and not on the summary chronogram.

Alternatively, one could try to choose the \enquote{best} chronogram from a set of possible evolutionary hypotheses. Several characteristics of the data used for dating analyses as well as from the output chronogram itself, could be used to score quality of source chronograms. Some characteristics that are often cited in published studies as a measure of improved age estimates as compared to previously published estimates are: quality of alignment (missing data, GC content), lineage sampling (strategy and proportion), phylogenetic and dating inference method, number of fossils used as calibrations, support for nodes and ages, and magnitude of confidence intervals. To facilitate subsetting of source chronograms following different criteria by the users, this information should be included as metadata manually entered by curators in the future.

In other areas of biological research, such as ecology and conservation biology, it has been shown that at least some data on lineage divergence represents a relevant improvement for testing alternative hypothesis using phylogenetic distance (Webb, Ackerly, \& Kembel, 2008). Hence, we integrated into datelife's workflow different ways of creating branch lengths in the absence of starting branch length information for taxa lacking this information (BLADJ option). Making up branch lengths in this or other ways is accepted in scientific publications: Jetz, Thomas, Joy, Hartmann, and Mooers (2012), created a time-calibrated tree of all 9,993 bird species, where 67\% had molecular data and the rest was simulated; Rabosky et al. (2018) created a time-calibrated tree of 31,536 ray-finned fishes, of which only 37\% had molecular data; Smith and Brown (2018) constructed a tree of 353,185 seed plants where only 23\% had molecular data. Taken to the extreme, one could make a fully resolved, calibrated tree of all modern and extinct taxa using a single taxonomy and a single calibration with the polytomy resolution and branch imputation methods. There has yet to be a thorough analysis of what can go wrong when one goes beyond the data in this way, so we urge caution; we also urge readers to follow the example of many of the large tree papers cited above and make sure results are substantially similar between trees fully reconstructed with molecular or other data, and trees that are reconstructed using taxonomy by resolving polytomies at random following a statistical model.

\hypertarget{conclusions}{%
\section{Conclusions}\label{conclusions}}

Divergence time information is key to many areas of evolutionary studies: trait evolution,
diversification, biogeography, macroecology and more. It is also crucial for science communication and education, but generating chronograms \emph{de novo} is difficult,
especially for those who want to use phylogenies but who are not systematists, or
do not have the time to acquire and develop the necessary knowledge and data curation skills. Moreover, years of primarily public funded research have resulted in vast amounts of chronograms that are already available on scientific publications, but hidden to the public and scientific community for reuse.

\texttt{datelife} allows easy and fast summarization of publicly available information
on time of lineage divergence. This provides a straightforward way to get an informed idea on the state of knowledge of the time frame of evolution of different regions of the tree of life, and allows identification of regions that require more research or that have conflicting information.
Both summary and newly generated trees are useful to evaluate evolutionary hypotheses in different areas of research. \texttt{datelife} helps with awareness of the existing variation in expert time of divergence data, and will foster exploration of the effect of alternative divergence time hypothesis on the results of analyses, nurturing a culture of more cautious interpretation of evolutionary results.

\hypertarget{availability}{%
\section{Availability}\label{availability}}

\texttt{datelife} is free and open source and it can be used through its current website
\url{http://www.datelife.org/query/}, through its R package, and through Phylotastic's project web portal \url{http://phylo.cs.nmsu.edu:3000/}.
\texttt{datelife}'s website is maintained using RStudio's shiny server and the shiny package open infrastructure, as well as Docker.
\texttt{datelife}'s R package stable version will be available
for installation from the CRAN repository (\url{https://cran.r-project.org/package=datelife})
using the command \texttt{install.packages(pkgs\ =\ "datelife")} from within R. Development versions
are available from the GitHub repository (\url{https://github.com/phylotastic/datelife})
and can be installed using the command \texttt{devtools::install\_github("phylotastic/datelife")}.

\hypertarget{supplementary-material}{%
\section{Supplementary Material}\label{supplementary-material}}

Code used to generate all versions of this manuscript, the biological examples, as well as the benchmark of functionalities are available at \href{https://github.com/LunaSare/datelifeMS1}{datelifeMS1}, \href{https://github.com/LunaSare/datelife_examples}{datelife\_examples}, and \href{https://github.com/LunaSare/datelife_benchmark}{datelife\_benchmark} repositories in LLSR's GitHub account.

\hypertarget{funding}{%
\section{Funding}\label{funding}}

Funding was provided by the US National Science Foundation (NSF) grants ABI-1458603 to Datelife project and DBI-0905606 to the National Evolutionary Synthesis Center (NESCent), and the Phylotastic project Grant ABI-1458572.

\hypertarget{acknowledgements}{%
\section{Acknowledgements}\label{acknowledgements}}

We thank colleagues from the O'Meara Lab at the University
of Tennesse Knoxville for suggestions, discussions and software testing.
The late National Evolutionary Synthesis Center (NESCent), which sponsored hackathons
that led to initial work on this project. The team that assembled \texttt{datelife}'s first proof of concept: Tracy Heath, Jonathan Eastman, Peter Midford, Joseph Brown, Matt Pennell, Mike Alfaro, and Luke Harmon.
The Open Tree of Life project that provides the open, metadata rich repository of
trees used for \texttt{datelife}.
The many scientists who publish their chronograms in an open, reusable form, and
the scientists who curate them for deposition in the Open Tree of Life repository.
The NSF for funding nearly all the above, in addition to the ABI grant that funded this project itself.

\newpage

\hypertarget{references}{%
\section{References}\label{references}}

\begingroup
\setlength{\parindent}{-0.5in}
\setlength{\leftskip}{0.5in}

\hypertarget{refs}{}
\leavevmode\hypertarget{ref-antonelli2017supersmart}{}%
Antonelli, A., Hettling, H., Condamine, F. L., Vos, K., Nilsson, R. H., Sanderson, M. J., \ldots{} Vos, R. A. (2017). Toward a self-updating platform for estimating rates of speciation and migration, ages, and relationships of Taxa. \emph{Systematic Biology}, \emph{66}(2), 153--166. \url{https://doi.org/10.1093/sysbio/syw066}

\leavevmode\hypertarget{ref-Bapst2012a}{}%
Bapst, D. W. (2012). Paleotree: An R package for paleontological and phylogenetic analyses of evolution. \emph{Methods in Ecology and Evolution}, \emph{3}(5), 803--807. \url{https://doi.org/10.1111/j.2041-210X.2012.00223.x}

\leavevmode\hypertarget{ref-barba2018constraining}{}%
Barba-Montoya, J., Reis, M. dos, Schneider, H., Donoghue, P. C., \& Yang, Z. (2018). Constraining uncertainty in the timescale of angiosperm evolution and the veracity of a cretaceous terrestrial revolution. \emph{New Phytologist}, \emph{218}(2), 819--834.

\leavevmode\hypertarget{ref-barker2012going}{}%
Barker, F. K., Burns, K. J., Klicka, J., Lanyon, S. M., \& Lovette, I. J. (2012). Going to extremes: Contrasting rates of diversification in a recent radiation of new world passerine birds. \emph{Systematic Biology}, \emph{62}(2), 298--320.

\leavevmode\hypertarget{ref-barker2015new}{}%
Barker, F. K., Burns, K. J., Klicka, J., Lanyon, S. M., \& Lovette, I. J. (2015). New insights into new world biogeography: An integrated view from the phylogeny of blackbirds, cardinals, sparrows, tanagers, warblers, and allies. \emph{The Auk: Ornithological Advances}, \emph{132}(2), 333--348.

\leavevmode\hypertarget{ref-Boyle2013}{}%
Boyle, B., Hopkins, N., Lu, Z., Raygoza Garay, J. A., Mozzherin, D., Rees, T., \ldots{} Enquist, B. J. (2013). The taxonomic name resolution service: An online tool for automated standardization of plant names. \emph{BMC Bioinformatics}, \emph{14}(1). \url{https://doi.org/10.1186/1471-2105-14-16}

\leavevmode\hypertarget{ref-Britton2007}{}%
Britton, T., Anderson, C. L., Jacquet, D., Lundqvist, S., \& Bremer, K. (2007). Estimating Divergence Times in Large Phylogenetic Trees. \emph{Systematic Biology}, \emph{56}(788777878), 741--752. \url{https://doi.org/10.1080/10635150701613783}

\leavevmode\hypertarget{ref-burns2014phylogenetics}{}%
Burns, K. J., Shultz, A. J., Title, P. O., Mason, N. A., Barker, F. K., Klicka, J., \ldots{} Lovette, I. J. (2014). Phylogenetics and diversification of tanagers (passeriformes: Thraupidae), the largest radiation of neotropical songbirds. \emph{Molecular Phylogenetics and Evolution}, \emph{75}, 41--77.

\leavevmode\hypertarget{ref-Chamberlain2018}{}%
Chamberlain, S. (2018). \emph{bold: Interface to Bold Systems API}. Retrieved from \url{https://CRAN.R-project.org/package=bold}

\leavevmode\hypertarget{ref-Chamberlain2013}{}%
Chamberlain, S. A., \& Szöcs, E. (2013). taxize : taxonomic search and retrieval in R {[}version 2; referees: 3 approved{]}. \emph{F1000Research}, \emph{2}(191), 1--29. \url{https://doi.org/10.12688/f1000research.2-191.v2}

\leavevmode\hypertarget{ref-claramunt2015new}{}%
Claramunt, S., \& Cracraft, J. (2015). A new time tree reveals earth history's imprint on the evolution of modern birds. \emph{Science Advances}, \emph{1}(11), e1501005.

\leavevmode\hypertarget{ref-gibb2015new}{}%
Gibb, G. C., England, R., Hartig, G., McLenachan, P. A., Taylor Smith, B. L., McComish, B. J., \ldots{} Penny, D. (2015). New zealand passerines help clarify the diversification of major songbird lineages during the oligocene. \emph{Genome Biology and Evolution}, \emph{7}(11), 2983--2995.

\leavevmode\hypertarget{ref-Harmon2008}{}%
Harmon, L., Weir, J., Brock, C., Glor, R., \& Challenger, W. (2008). GEIGER: investigating evolutionary radiations. \emph{Bioinformatics}, \emph{24}, 129--131.

\leavevmode\hypertarget{ref-Hedges2015}{}%
Hedges, S. B., Marin, J., Suleski, M., Paymer, M., \& Kumar, S. (2015). Tree of life reveals clock-like speciation and diversification. \emph{Molecular Biology and Evolution}, \emph{32}(4), 835--845. \url{https://doi.org/10.1093/molbev/msv037}

\leavevmode\hypertarget{ref-Heibl2008}{}%
Heibl, C. (2008). \emph{PHYLOCH: R language tree plotting tools and interfaces to diverse phylogenetic software packages.} Retrieved from \url{http://www.christophheibl.de/Rpackages.html}

\leavevmode\hypertarget{ref-hooper2017chromosomal}{}%
Hooper, D. M., \& Price, T. D. (2017). Chromosomal inversion differences correlate with range overlap in passerine birds. \emph{Nature Ecology \& Evolution}, \emph{1}(10), 1526.

\leavevmode\hypertarget{ref-Jetz2012}{}%
Jetz, W., Thomas, G., Joy, J. J. B., Hartmann, K., \& Mooers, A. (2012). The global diversity of birds in space and time. \emph{Nature}, \emph{491}(7424), 444--448. \url{https://doi.org/10.1038/nature11631}

\leavevmode\hypertarget{ref-Kumar2017}{}%
Kumar, S., Stecher, G., Suleski, M., \& Hedges, S. B. (2017). TimeTree: A Resource for Timelines, Timetrees, and Divergence Times. \emph{Molecular Biology and Evolution}, \emph{34}(7), 1812--1819. \url{https://doi.org/10.1093/molbev/msx116}

\leavevmode\hypertarget{ref-magallon2015metacalibrated}{}%
Magallón, S., Gómez-Acevedo, S., Sánchez-Reyes, L. L., \& Hernández-Hernández, T. (2015). A metacalibrated time-tree documents the early rise of flowering plant phylogenetic diversity. \emph{New Phytologist}, \emph{207}(2), 437--453.

\leavevmode\hypertarget{ref-mctavish2015phylesystem}{}%
McTavish, E. J., Hinchliff, C. E., Allman, J. F., Brown, J. W., Cranston, K. A., Holder, M. T., \ldots{} Smith, S. A. (2015). Phylesystem: A git-based data store for community-curated phylogenetic estimates. \emph{Bioinformatics}, \emph{31}(17), 2794--2800.

\leavevmode\hypertarget{ref-Michonneau2016}{}%
Michonneau, F., Brown, J. W., \& Winter, D. J. (2016). rotl: an R package to interact with the Open Tree of Life data. \emph{Methods in Ecology and Evolution}, \emph{7}(12), 1476--1481. \url{https://doi.org/10.1111/2041-210X.12593}

\leavevmode\hypertarget{ref-Ooms2018}{}%
Ooms, J., \& Chamberlain, S. (2018). \emph{Phylocomr: Interface to 'phylocom'}. Retrieved from \url{https://CRAN.R-project.org/package=phylocomr}

\leavevmode\hypertarget{ref-Paradis2004}{}%
Paradis, E., Claude, J., \& Strimmer, K. (2004). APE: analyses of phylogenetics and evolution in R language. \emph{Bioinformatics}, \emph{20}, 289--290.

\leavevmode\hypertarget{ref-price2014niche}{}%
Price, T. D., Hooper, D. M., Buchanan, C. D., Johansson, U. S., Tietze, D. T., Alström, P., \ldots{} others. (2014). Niche filling slows the diversification of himalayan songbirds. \emph{Nature}, \emph{509}(7499), 222.

\leavevmode\hypertarget{ref-rabosky2018inverse}{}%
Rabosky, D. L., Chang, J., Title, P. O., Cowman, P. F., Sallan, L., Friedman, M., \ldots{} others. (2018). An inverse latitudinal gradient in speciation rate for marine fishes. \emph{Nature}, \emph{559}(7714), 392.

\leavevmode\hypertarget{ref-ramshaw1972time}{}%
Ramshaw, J., Richardson, D., Meatyard, B., Brown, R., Richardson, M., Thompson, E., \& Boulter, D. (1972). The time of origin of the flowering plants determined by using amino acid sequence data of cytochrome c. \emph{New Phytologist}, \emph{71}(5), 773--779.

\leavevmode\hypertarget{ref-RCoreTeam2018}{}%
R Core Team. (2018). \emph{R: a language and environment for statistical computing}. Vienna, Austria: R Foundation for Statistical Computing.

\leavevmode\hypertarget{ref-Revell2012}{}%
Revell, L. J. (2012). Phytools: An r package for phylogenetic comparative biology (and other things). \emph{Methods in Ecology and Evolution}, \emph{3}, 217--223.

\leavevmode\hypertarget{ref-sanderson2003r8s}{}%
Sanderson, M. J. (2003). R8s: Inferring absolute rates of molecular evolution and divergence times in the absence of a molecular clock. \emph{Bioinformatics}, \emph{19}(2), 301--302.

\leavevmode\hypertarget{ref-sanderson2001sources}{}%
Sanderson, M. J., \& Doyle, J. A. (2001). Sources of error and confidence intervals in estimating the age of angiosperms from rbcL and 18S rDNA data. \emph{American Journal of Botany}, \emph{88}(8), 1499--1516.

\leavevmode\hypertarget{ref-schenk2016sec}{}%
Schenk, J. J. (2016). Consequences of secondary calibrations on divergence time estimates. \emph{PLoS ONE}, \emph{11}(1). \url{https://doi.org/10.1371/journal.pone.0148228}

\leavevmode\hypertarget{ref-smith2018constructing}{}%
Smith, S. A., \& Brown, J. W. (2018). Constructing a broadly inclusive seed plant phylogeny. \emph{American Journal of Botany}, \emph{105}(3), 302--314.

\leavevmode\hypertarget{ref-title2016macrophylogenies}{}%
Title, P. O., \& Rabosky, D. L. (2016). Do Macrophylogenies Yield Stable Macroevolutionary Inferences? An Example from Squamate Reptiles. \emph{Systematic Biology}, syw102. \url{https://doi.org/10.1093/sysbio/syw102}

\leavevmode\hypertarget{ref-Webb2008}{}%
Webb, C. O., Ackerly, D. D., \& Kembel, S. W. (2008). Phylocom: Software for the analysis of phylogenetic community structure and trait evolution. \emph{Bioinformatics}, \emph{24}(18), 2098--2100. \url{https://doi.org/10.1093/bioinformatics/btn358}

\endgroup


\end{document}
