% Options for packages loaded elsewhere
\PassOptionsToPackage{unicode}{hyperref}
\PassOptionsToPackage{hyphens}{url}
%
\documentclass[
  english,
  man]{apa6}
\usepackage{lmodern}
\usepackage{amssymb,amsmath}
\usepackage{ifxetex,ifluatex}
\ifnum 0\ifxetex 1\fi\ifluatex 1\fi=0 % if pdftex
  \usepackage[T1]{fontenc}
  \usepackage[utf8]{inputenc}
  \usepackage{textcomp} % provide euro and other symbols
\else % if luatex or xetex
  \usepackage{unicode-math}
  \defaultfontfeatures{Scale=MatchLowercase}
  \defaultfontfeatures[\rmfamily]{Ligatures=TeX,Scale=1}
\fi
% Use upquote if available, for straight quotes in verbatim environments
\IfFileExists{upquote.sty}{\usepackage{upquote}}{}
\IfFileExists{microtype.sty}{% use microtype if available
  \usepackage[]{microtype}
  \UseMicrotypeSet[protrusion]{basicmath} % disable protrusion for tt fonts
}{}
\makeatletter
\@ifundefined{KOMAClassName}{% if non-KOMA class
  \IfFileExists{parskip.sty}{%
    \usepackage{parskip}
  }{% else
    \setlength{\parindent}{0pt}
    \setlength{\parskip}{6pt plus 2pt minus 1pt}}
}{% if KOMA class
  \KOMAoptions{parskip=half}}
\makeatother
\usepackage{xcolor}
\IfFileExists{xurl.sty}{\usepackage{xurl}}{} % add URL line breaks if available
\IfFileExists{bookmark.sty}{\usepackage{bookmark}}{\usepackage{hyperref}}
\hypersetup{
  pdftitle={DateLife: leveraging databases and analytical tools to reveal the dated Tree of Life},
  pdfauthor={Luna L. Sánchez Reyes1,2, Emily Jane McTavish1, \& Brian O'Meara2},
  pdflang={en-EN},
  pdfkeywords={Tree; Phylogeny; Scaling; Dating; Ages; Divergence times; Open Science; Congruification; Supertree; Calibrations},
  hidelinks,
  pdfcreator={LaTeX via pandoc}}
\urlstyle{same} % disable monospaced font for URLs
\usepackage{graphicx,grffile}
\makeatletter
\def\maxwidth{\ifdim\Gin@nat@width>\linewidth\linewidth\else\Gin@nat@width\fi}
\def\maxheight{\ifdim\Gin@nat@height>\textheight\textheight\else\Gin@nat@height\fi}
\makeatother
% Scale images if necessary, so that they will not overflow the page
% margins by default, and it is still possible to overwrite the defaults
% using explicit options in \includegraphics[width, height, ...]{}
\setkeys{Gin}{width=\maxwidth,height=\maxheight,keepaspectratio}
% Set default figure placement to htbp
\makeatletter
\def\fps@figure{htbp}
\makeatother
\setlength{\emergencystretch}{3em} % prevent overfull lines
\providecommand{\tightlist}{%
  \setlength{\itemsep}{0pt}\setlength{\parskip}{0pt}}
\setcounter{secnumdepth}{-\maxdimen} % remove section numbering
% Make \paragraph and \subparagraph free-standing
\ifx\paragraph\undefined\else
  \let\oldparagraph\paragraph
  \renewcommand{\paragraph}[1]{\oldparagraph{#1}\mbox{}}
\fi
\ifx\subparagraph\undefined\else
  \let\oldsubparagraph\subparagraph
  \renewcommand{\subparagraph}[1]{\oldsubparagraph{#1}\mbox{}}
\fi
% Manuscript styling
\usepackage{upgreek}
\captionsetup{font=singlespacing,justification=justified}

% Table formatting
\usepackage{longtable}
\usepackage{lscape}
% \usepackage[counterclockwise]{rotating}   % Landscape page setup for large tables
\usepackage{multirow}		% Table styling
\usepackage{tabularx}		% Control Column width
\usepackage[flushleft]{threeparttable}	% Allows for three part tables with a specified notes section
\usepackage{threeparttablex}            % Lets threeparttable work with longtable

% Create new environments so endfloat can handle them
% \newenvironment{ltable}
%   {\begin{landscape}\centering\begin{threeparttable}}
%   {\end{threeparttable}\end{landscape}}
\newenvironment{lltable}{\begin{landscape}\centering\begin{ThreePartTable}}{\end{ThreePartTable}\end{landscape}}

% Enables adjusting longtable caption width to table width
% Solution found at http://golatex.de/longtable-mit-caption-so-breit-wie-die-tabelle-t15767.html
\makeatletter
\newcommand\LastLTentrywidth{1em}
\newlength\longtablewidth
\setlength{\longtablewidth}{1in}
\newcommand{\getlongtablewidth}{\begingroup \ifcsname LT@\roman{LT@tables}\endcsname \global\longtablewidth=0pt \renewcommand{\LT@entry}[2]{\global\advance\longtablewidth by ##2\relax\gdef\LastLTentrywidth{##2}}\@nameuse{LT@\roman{LT@tables}} \fi \endgroup}

% \setlength{\parindent}{0.5in}
% \setlength{\parskip}{0pt plus 0pt minus 0pt}

% \usepackage{etoolbox}
\makeatletter
\patchcmd{\HyOrg@maketitle}
  {\section{\normalfont\normalsize\abstractname}}
  {\section*{\normalfont\normalsize\abstractname}}
  {}{\typeout{Failed to patch abstract.}}
\patchcmd{\HyOrg@maketitle}
  {\section{\protect\normalfont{\@title}}}
  {\section*{\protect\normalfont{\@title}}}
  {}{\typeout{Failed to patch title.}}
\makeatother
\shorttitle{DateLife: revealing the dated Tree of Life}
\keywords{Tree; Phylogeny; Scaling; Dating; Ages; Divergence times; Open Science; Congruification; Supertree; Calibrations\newline\indent Word count: 55}
\DeclareDelayedFloatFlavor{ThreePartTable}{table}
\DeclareDelayedFloatFlavor{lltable}{table}
\DeclareDelayedFloatFlavor*{longtable}{table}
\makeatletter
\renewcommand{\efloat@iwrite}[1]{\immediate\expandafter\protected@write\csname efloat@post#1\endcsname{}}
\makeatother
\usepackage{lineno}

\linenumbers
\usepackage{csquotes}
\ifxetex
  % Load polyglossia as late as possible: uses bidi with RTL langages (e.g. Hebrew, Arabic)
  \usepackage{polyglossia}
  \setmainlanguage[]{english}
\else
  \usepackage[shorthands=off,main=english]{babel}
\fi

\title{DateLife: leveraging databases and analytical tools to reveal the dated Tree of Life}
\author{Luna L. Sánchez Reyes\textsuperscript{1,2}, Emily Jane McTavish\textsuperscript{1}, \& Brian O'Meara\textsuperscript{2}}
\date{}


\authornote{

School of Natural Sciences, University of California, Merced, Science and Engineering Building 1.

Department of Ecology and Evolutionary Biology, University of Tennessee, Knoxville, 425 Hesler Biology Building, Knoxville, TN 37996, USA.

The authors made the following contributions. Luna L. Sánchez Reyes: Data curation, Investigation, Software, Visualization, Validation, Writing - Original Draft Preparation, Writing - Review \& Editing; Emily Jane McTavish: Resources, Software, Writing - Review \& Editing; Brian O'Meara: Conceptualization, Funding acquisition, Methodology, Resources, Software, Supervision, Writing - Review \& Editing.

Correspondence concerning this article should be addressed to Luna L. Sánchez Reyes, . E-mail: \href{mailto:sanchez.reyes.luna@gmail.com}{\nolinkurl{sanchez.reyes.luna@gmail.com}}

}

\affiliation{\vspace{0.5cm}\textsuperscript{1} University of California, Merced\\\textsuperscript{2} University of Tennessee, Knoxville}

\abstract{
Time of evolutionary origin is fundamental for understanding biological processes.

The combination of new analytical techniques, availability of more fossil and molecular
data, and efforts to improve data sharing practices in biology has resulted in a steady accumulation
of time of lineage divergence in public and open databases such as TreeBASE, Dryad, and Open Tree
of Life for a large quantity and diversity of organisms in the last few decades.

However, getting a tree with branch lengths proportional to time remains difficult for many biologists and the non-academic community, despite its importance for biological research, medicine, education, and science communication.

Here we present \texttt{datelife}, a service implemented via an R package and a web site
(\url{http://www.datelife.org/}) for efficient reuse, summary and reanalysis of expert, peer-reviewed, public data on time of lineage divergence.

Main results:
1. blah
2. blah
3. blah

Results in a general context:
All source and summary chronograms can be saved in formats that permit easy reuse and reanalysis. Summary and newly generated trees are potentially useful to evaluate evolutionary hypothesis in different areas of research in biology.
How well this trees work for this purpose still needs to be tested.

\texttt{datelife} will be useful to increase awereness on the existing variation in expert time of divergence data, and might foster exploration of the effect of alternative divergence time hypothesis on the results of analyses, nurturing a culture of more cautious interpretation of evolutionary results.
}



\begin{document}
\maketitle

\begin{verbatim}
## Warning in data(opentree_chronograms): data set 'opentree_chronograms' not found
\end{verbatim}

\hypertarget{introduction}{%
\section{Introduction}\label{introduction}}

\hypertarget{description}{%
\section{Description}\label{description}}

We report how we determined our sample size, all data exclusions (if any), all manipulations, and all measures in the study.

\hypertarget{benchmark}{%
\section{Benchmark}\label{benchmark}}

\hypertarget{example}{%
\section{Example}\label{example}}

We used R (Version 4.1.0; R Core Team, 2021) and the R-package \emph{papaja} (Version 0.1.0.9997; Aust \& Barth, 2020) for all our analyses.

\hypertarget{discussion}{%
\section{Discussion}\label{discussion}}

\hypertarget{conclusions}{%
\section{Conclusions}\label{conclusions}}

\hypertarget{availability}{%
\section{Availability}\label{availability}}

\texttt{datelife} is free and open source and it can be used through its current website
\url{http://www.datelife.org/query/}, through its R package, and through Phylotastic's project web portal \url{http://phylo.cs.nmsu.edu:3000/}.
\texttt{datelife}'s website is maintained using RStudio's shiny server and the shiny package open infrastructure, as well as Docker.
\texttt{datelife}'s R package stable version will be available
for installation from the CRAN repository (\url{https://cran.r-project.org/package=datelife})
using the command \texttt{install.packages(pkgs\ =\ "datelife")} from within R. Development versions
are available from the GitHub repository (\url{https://github.com/phylotastic/datelife})
and can be installed using the command \texttt{devtools::install\_github("phylotastic/datelife")}.

\hypertarget{supplementary-material}{%
\section{Supplementary Material}\label{supplementary-material}}

Code used to generate all versions of this manuscript, the biological examples, as well as the benchmark of functionalities are available at \href{https://github.com/LunaSare/datelifeMS1}{datelifeMS1}, \href{https://github.com/LunaSare/datelife_examples}{datelife\_examples}, and \href{https://github.com/LunaSare/datelife_benchmark}{datelife\_benchmark} repositories in LLSR's GitHub account.

\hypertarget{funding}{%
\section{Funding}\label{funding}}

Funding was provided by the US National Science Foundation (NSF) grants ABI-1458603 to Datelife project and DBI-0905606 to the National Evolutionary Synthesis Center (NESCent), and the Phylotastic project Grant ABI-1458572.

\hypertarget{acknowledgements}{%
\section{Acknowledgements}\label{acknowledgements}}

We thank colleagues from the O'Meara Lab at the University
of Tennesse Knoxville for suggestions, discussions and software testing.
The late National Evolutionary Synthesis Center (NESCent), which sponsored hackathons
that led to initial work on this project. The team that assembled \texttt{datelife}'s first proof of concept: Tracy Heath, Jonathan Eastman, Peter Midford, Joseph Brown, Matt Pennell, Mike Alfaro, and Luke Harmon.
The Open Tree of Life project that provides the open, metadata rich repository of
trees used for \texttt{datelife}.
The many scientists who publish their chronograms in an open, reusable form, and
the scientists who curate them for deposition in the Open Tree of Life repository.
The NSF for funding nearly all the above, in addition to the ABI grant that funded this project itself.

\newpage

\hypertarget{references}{%
\section{References}\label{references}}

\begingroup
\setlength{\parindent}{-0.5in}
\setlength{\leftskip}{0.5in}

\hypertarget{refs}{}
\leavevmode\hypertarget{ref-R-papaja}{}%
Aust, F., \& Barth, M. (2020). \emph{papaja: Create APA manuscripts with R Markdown}. Retrieved from \url{https://github.com/crsh/papaja}

\leavevmode\hypertarget{ref-barker2012going}{}%
Barker, F. K., Burns, K. J., Klicka, J., Lanyon, S. M., \& Lovette, I. J. (2012). Going to extremes: Contrasting rates of diversification in a recent radiation of new world passerine birds. \emph{Systematic Biology}, \emph{62}(2), 298--320.

\leavevmode\hypertarget{ref-barker2015new}{}%
Barker, F. K., Burns, K. J., Klicka, J., Lanyon, S. M., \& Lovette, I. J. (2015). New insights into new world biogeography: An integrated view from the phylogeny of blackbirds, cardinals, sparrows, tanagers, warblers, and allies. \emph{The Auk: Ornithological Advances}, \emph{132}(2), 333--348.

\leavevmode\hypertarget{ref-burns2014phylogenetics}{}%
Burns, K. J., Shultz, A. J., Title, P. O., Mason, N. A., Barker, F. K., Klicka, J., \ldots{} Lovette, I. J. (2014). Phylogenetics and diversification of tanagers (passeriformes: Thraupidae), the largest radiation of neotropical songbirds. \emph{Molecular Phylogenetics and Evolution}, \emph{75}, 41--77.

\leavevmode\hypertarget{ref-claramunt2015new}{}%
Claramunt, S., \& Cracraft, J. (2015). A new time tree reveals earth history's imprint on the evolution of modern birds. \emph{Science Advances}, \emph{1}(11), e1501005.

\leavevmode\hypertarget{ref-gibb2015new}{}%
Gibb, G. C., England, R., Hartig, G., McLenachan, P. A., Taylor Smith, B. L., McComish, B. J., \ldots{} Penny, D. (2015). New zealand passerines help clarify the diversification of major songbird lineages during the oligocene. \emph{Genome Biology and Evolution}, \emph{7}(11), 2983--2995.

\leavevmode\hypertarget{ref-Hedges2015}{}%
Hedges, S. B., Marin, J., Suleski, M., Paymer, M., \& Kumar, S. (2015). Tree of life reveals clock-like speciation and diversification. \emph{Molecular Biology and Evolution}, \emph{32}(4), 835--845. \url{https://doi.org/10.1093/molbev/msv037}

\leavevmode\hypertarget{ref-hooper2017chromosomal}{}%
Hooper, D. M., \& Price, T. D. (2017). Chromosomal inversion differences correlate with range overlap in passerine birds. \emph{Nature Ecology \& Evolution}, \emph{1}(10), 1526.

\leavevmode\hypertarget{ref-Jetz2012}{}%
Jetz, W., Thomas, G., Joy, J. J. B., Hartmann, K., \& Mooers, A. (2012). The global diversity of birds in space and time. \emph{Nature}, \emph{491}(7424), 444--448. \url{https://doi.org/10.1038/nature11631}

\leavevmode\hypertarget{ref-price2014niche}{}%
Price, T. D., Hooper, D. M., Buchanan, C. D., Johansson, U. S., Tietze, D. T., Alström, P., \ldots{} others. (2014). Niche filling slows the diversification of himalayan songbirds. \emph{Nature}, \emph{509}(7499), 222.

\leavevmode\hypertarget{ref-R-base}{}%
R Core Team. (2021). \emph{R: A language and environment for statistical computing}. Vienna, Austria: R Foundation for Statistical Computing. Retrieved from \url{https://www.R-project.org/}

\endgroup


\end{document}
